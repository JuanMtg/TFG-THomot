\thispagestyle{plain}
\addcontentsline{toc}{chapter}{Resumen} 

\begin{abstract}
En este trabajo se estudian herramientas básicas de la Teoría de Homotopía. Primero se introducen brevemente conceptos básicos sobre homotopía, para proceder a introducir la suspensión $\Sigma X$ de un espacio y el espacio de lazos $\Omega X$. Se continúa con fibraciones y cofibraciones, viendo resultados y ejemplos sobre estos conceptos. Finalmente, se estudian los grupos de homotopía de orden superior y se relacionan con lo anteriormente visto.
\end{abstract}

\selectlanguage{english}
\addcontentsline{toc}{chapter}{Abstract} 

\begin{abstract}
In this paper we study the basic tools in Homotopy Theory. First, we briefly introduce basic concepts in homotopy theory. Then we define the suspension $\Sigma X$ and the loop space $\Omega X$ of a given space $X$. To follow up, fibrations and cofibrations are introduced along with some examples. Finally, we study higher order homotopy groups and stablish a connection between them and the concepts previously defined.
\end{abstract}

\selectlanguage{spanish}