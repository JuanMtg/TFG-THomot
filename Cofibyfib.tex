\chapter{Fibraciones y Cofibraciones}
En esta parte introduciremos los conceptos de fibraciones y cofibraciones, relacionándolas con lo anteriormente visto.
\section{Cofibraciones}
\begin{defin}
Sea $A \subset X$ un subespacio. Decimos que $i : A \longhookrightarrow X$ o el par $(X, A)$ tiene la propiedad de extensión homotópica (HEP) con respecto al espacio $Y$ si dada una homotopía
\[ G : A \times I \longrightarrow Y \]
que tiene una extensión a X en su inicio, esto es, si existe $f: X \longrightarrow Y $ tal que $f(a) = G(a, 0)$ $\forall \ a \in A$, entonces tienen una extensión a toda ella a $X \times I$, es decir, existe $F : X \times I \longrightarrow Y$ tal que $f(x) = F(x,0)$ y $F(a,t) = G(a, t)$ $\forall \ t \in I, \ a \in A$ \par 
De forma equivalente, $(X,A)$ tiene la HEP si el siguiente diagrama puede completarse con el morfismo punteado:
\[
\begin{tikzcd}
	{} 						  & X \drar{i_0} \arrow[bend left]{drr}{f}			   &        					&   \\
	A \urar[hook] \drar{i_0}  &   												   & X \times I \rar[dashed]{F} & Y \\
	   						  & A \times I  \urar[hook] \arrow[bend right]{urr}{G} &   							&
\end{tikzcd}
\]
\end{defin}
\begin{defin}
Decimos que un par $(X,A)$ ( o la inclusión $i: A \longhookrightarrow X$) es una cofibración si posee la HEP con respecto a todo espacio Y. \par 
Es decir, $A \subset X$ es cofibración si toda aplicación  
\[ f : X \times \{0\} \cup A \times I \longrightarrow Y \]
se extiende a $X \times I$.
O de forma gráfica, \nts{Insertar imagen p 35}
\end{defin}

\textbf{Observaciones:}
\begin{enumerate}
\item No todas las inclusiones $A \subset X$ son cofibraciones. Un contraejemplo de esto es el espacio formado por los segmentos de $(0, 0)$ en $(1, \frac{1}{n})$.
Esto es, $ A = \{0\} \times [0,1] \cup [0,1] \times \{0\} \bigcup_{n \in \bb{N}} \{ (t, \frac{t}{n}) : t \in [0,1] \} $ \nts{Insertar imagen p35} \\
$A$ no es un retracto de $I^2$, luego no se tiene la extensión para la identidad 
\[ 
\begin{tikzcd}
A \arrow{rr}{Id} \drar[hook] & & A \\
 & X \urar[dashed] &
\end{tikzcd}
\]
Se tiene un caso análogo para el espacio peine anteriormente visto.

\item Podríamos haber definido la cofibración sustituyendo la inclusión por una aplicación $f : A \longrightarrow X $ cualquiera. Podemos demostrar entonces que $f$ es inyectiva y pero no es necesariamente un homeomorfismo sobre su imagen. Si suponemos que el par $(X, A)$ es una cofibración y que $X$ es Hausdorff, entonces se puede probar que $A$ es cerrado, teniendo así el homeomorfismo sobre su imagen. Desde este momento, se supondrá siempre esto, ya que el caso más general rara vez es necesario.
\end{enumerate}

\begin{teor}
El par $(X, A)$ es una cofibración si y sólo si $X \times \{0\} \cup A \times I$ es un retracto de $X \times I$.
\end{teor}
\begin{demo}
Para la implicación directa, como $(X,A)$ es una cofibración, tomando entonces la identidad 
\[ X \times \{0\} \cup A \times I \stackrel{Id}{\longrightarrow} X \times \{0\} \cup A \times I \]
se extiende como el siguiente diagrama
\[
\begin{tikzcd}
	X \times \{0\} \cup A \times I \arrow{rr}{1} \drar[hook] & & X \times \{0\} \cup A \times I \\
				& X \times I \urar[dashed]{r} &
\end{tikzcd}
\]
Por tanto, $X \times I \stackrel{r}{\longrightarrow} X \times \{0\} \cup A \times I$ es una retracción. \par
Recíprocamente, al suponer que $A$ es cerrado, dadas dos aplicaciones cualesquiera 
\[ \begin{cases}
X \times \{0\} \longrightarrow Y \\
A \times I \longrightarrow Y
\end{cases} \]
que coinciden en $A \times \{0\}$, obtenemos una aplicación $X \times \{0\} \cup A \times I \longrightarrow Y$ que es continua al serlo en los cerrados. Componiendo esta aplicación con una retracción obtenemos la extensión que queríamos.
\end{demo}
También hay resultados que relacionan cofibraciones con CW-complejos.
\begin{teor}
SI $(X, A)$ es un par de CW-complejos, entonces $X \times \{0\} \cup A \times I$ es un retracto de deformación de $X \times I$.
\end{teor}
\begin{demo}
Existe una retracción $r : D^n \times I \longrightarrow D^n \times \{0\} \cup \partial D^n \times I$. \\
Tomando la homotopía 
\begin{align*}
H : (D^n \times I) \times I \longrightarrow D^n \times I \\
H(p, s) = s r(p) + (1-s)p
\end{align*}
induce una homotopía entre $Id_{D^n \times I} $ e $ir$. Por tanto, r es un retracto de deformación de $D^n \times I$. \\
Podemos ampliar este retracto de deformación a uno de $X^n \times I$ en $X^n \times \{0\} \cup (X^{n-1} \cup A^n) \times I$. \par
Como $X^n$ se obtiene adjuntando $n$-celdas a $X^{n-1} \cup A^n$, $X^n \times I$ se obtiene de $X^n \times \{0\} \cup (X^{n-1} \cup A^n) \times I$ adjuntando copias de $D^n \times I$ a lo largo de $D^n \times \{0\} \cup \partial D^n \times I$. \par 
SI aplicamos el retracto de deformación de $X^n \times I$ en $X^n \times \{0\} \cup (X^{n-1} \cup A^n) \times I$ en los intervalos ${\small \left[\faktor{1}{2^{n+1}}, \faktor{1}{2^n}\right]}$ y las concatenamos, esto es un retracto de deformación de $X \times I$ en $X \times \{0\} \cup A \times I$.
\end{demo}

\begin{teor}
SI el par $(X, A)$ es una cofibración y $A$ es contráctil, entonces $q : X \longrightarrow \faktor{X}{A}$ es una equivalenci de homotopía.
\end{teor}
\begin{demo}
Sea $H : X \times I \longrightarrow X$ una homotopía que extiende a la contracción $F : A \times I \longrightarrow A$ y que tiene como inicio a la identidad, $H(x,0) = x$ y verifica que $H(A,t) = F(A,t) \subset A \ \forall t$. La composición $ q \circ H_t : X \longrightarrow \faktor{X}{A}$ envía $A$ a un punto. Por tanto, factoriza como el diagrama
\[ \begin{tikzcd}
	X \rar{H_t} \dar{q} & X \dar{q} \\
	\faktor{X}{A} \rar{\overline{H}_t} & \faktor{X}{A}
\end{tikzcd} \]
Tomando $t = 1$ , entonces $H(A, 1) = F(A, 1) = \ast$ es el punto donde $A$ se contrae, luego existe una aplicación $g$ tal que el siguiente diagrama es conmutativo:
\[ \begin{tikzcd}
	X \rar{H_1} \dar{q} & X \dar{q} \\
	\faktor{X}{A} \rar{\overline{H}_1} \urar{g} & \faktor{X}{A}
\end{tikzcd} \]
En principio tenemos que conmuta el triángulo superior $gq = H_1$. Para el triángulo inferior, tenemos que $qg(\bar{x}) = qgq(x) = qH_1(x) =\overline{H}_1 q(x) = \overline{H}_1(\bar{x})  $ teniendo en cuenta que el primer diagrama conmuta para todo $t$. Por tanto, tenemos que $q$ y $g$ son inversos homotópicos
\begin{align*}
gq &= H_1 \simeq H_0 = Id_X \\
qg &= \overline{H}_1 \simeq \overline{H}_0 = Id_{\faktor{X}{A}}
\end{align*}
\end{demo}
Veamos ahora algunos ejemplos:
\begin{ejems}
\begin{enumerate}
\item \textbf{Grafos:} Los grafos que vemos a la izquierda son todos homotópicamente equivalentes. Se pueden ver que son retractos de $D^2 - \{x_0, x_1\}$, pero también podemos usar el teorema anterior. Tomando en los grafos $(A)$ y $(C)$ los segmentos rectos y haciendo cociente con respecto a ellos, obtenemos el grafo $(B)$. Así, por el teorema anterior, tenemos que son homotópicamente equivalentes. \par 
De forma más general, si $G$ es un grafo finito, toda arista con distintos finales se puede contraer a un punto. En este caso, toda componente de $G$ es homótopa o bien a un punto o a una suma puntual de $S^1$.

\item Consideremos $X$ el espacio formado por la unión de una $S^2$ y una $1$-celda por los polos. Por el teorema anterior, si identificamos los polos al mismo punto sobre la esfera o contraemos la $1$-celda a un punto, obtenemos dos espacios homotópicamente equivalentes. Así, tenemos que
\[ S^2 \cup e^1 = S^1 \vee S^2 = \faktor{S^2}{S^0} \]

\item \nts{Ejemplo unión de esferas}
\end{enumerate}
\end{ejems}

\subsection{Cómo pegar espacios en general}
Antes vimos el concepto de unir $n$-celdas a un espacio $X$. Ahora vamos a generalizar este concepto para cualesquiera espacios $X$ e $Y$. \par 
Para ello, consideramos un subespacio $A \subset Y $ y una aplicación $f : A \longrightarrow X$. El subespacio $A$ será el ``sitio por donde vamos a pegar'' los espacios, mientras que la aplicación es la forma de pegarlos. Una vez dados estos dos elementos, definimos:
\[ X \cup_f Y := \faktor{X \dot{\cup} Y}{x \in A \sim f(x)} \]
\nts{Insertar imagen p43} \\
Como hemos observado, pegar una celda es un caso particular, donde el subespacio que escogemos es la frontera de la celda. \par 
Otro caso interesante es el llamado cilindro de una aplicación. Sea $f: X \longrightarrow Y$ una aplicación continua. Definimos el cilindro de $f$, 
\[M_f = Y \cup_{\tilde{f}} (X \times I) \] 
donde
\begin{align*}
\tilde{f} : X \times I \longrightarrow Y \\
\tilde{f}(x,1) = f(x)
\end{align*}
esto es, $M_f = \faktor{Y \dot{\cup} (X \times I)}{(x,1) \sim f(x)}$

Si además colapsamos $X \times \{0\}$ a un punto, obtenemos el ``cono de $f$'', que también lo podemos definir como 
\[ C_f = \faktor{Y \dot{\cup} CX}{(x,1) \sim f(x)} \]
Visto esto, podemos factorizar la función $f$ como la siguiente composición: $f = p \circ i$, donde $i$ es la inclusión y $p$ es la aplicación que ``aplasta'' el cilindro al espacio $Y$, esto es $i(x) = (x,0)$  y $p : M_f \longrightarrow Y$ tal que $p \vert_Y = Id_Y$, $p(x,t) = f(x)$. \par 
Visto esto último, tenemos el siguiente resultado:
\begin{teor}
Sea $f : X \longrightarrow Y$ una aplicación y consideremos la factorización de $f$ dada por el cilindro de $f$, $f = p \circ i$. Entonces $i$ es una cofibración y $p$ una equivalencia de homotopía. En particular, toda aplicación $f$ es, salvo homotopía, una cofibración. \footnote{Mirar esta última parte}
\end{teor}
\begin{demo}
En efecto, $X \longhookrightarrow M_f$ es una cofibración, pues la retracción $I \times I \stackrel{r}{\longrightarrow} I \times \{0\} \cup \partial I \times I$ induce una retracción de $M_f \times I$ en $M_f \times \{0\} \cup X \times  I$. \par 
Por otra parte, la inclusión $j : Y \longrightarrow M_f$ es una equivalencia de homotopía inversa de $p$: $p \circ j = Id_Y$, $j \circ p \simeq Id_{M_f}$.
\end{demo}
En general, si $A \longhookrightarrow X$ es una cofibración, llamamos cofibra de A a $\faktor{X}{A}$, de forma que una cofibración es análogo a una sucesión exata corta
\[ A \longhookrightarrow X \longrightarrow \faktor{X}{A} \]
Si $f : X \longrightarrow Y$ es una aplicación cualquiera, la cofibra de $f$ es la cofibra de la cofibración asociada
\[ X \longrightarrow M_f \longrightarrow \faktor{M_f}{X} = C_f \]
que escribimos como: 
\[ X \stackrel{f}{\longrightarrow} Y \stackrel{q}{\longrightarrow} C_f  \]
donde $q = q' \circ j$ , $q' : M_f \longrightarrow C_f$. \par 
\nts{Insertar imagen p46}
que es una cofibración de cofibra $\Sigma X$. \par
Tenemos entonces una sucesión
\[ X \stackrel{f}{\longrightarrow} Y \stackrel{q}{\longrightarrow} C_f \stackrel{\delta}{\longrightarrow} \Sigma X \stackrel{\Sigma f}{\longrightarrow} \Sigma Y \stackrel{\Sigma q}{\longrightarrow} \Sigma C_f \longrightarrow \dots \]
que es la sucesión de ``Barratt-Puppe'' que da lugar a una sucesión exacta larga de grupos:
\[ [X, Z] \stackrel{f^*}{\longleftarrow} [Y, Z] \stackrel{q^*}{\longleftarrow} [C_f, Z] \stackrel{\delta^*}{\longleftarrow} [\Sigma X, Z] \stackrel{\Sigma f^*}{\longleftarrow} [\Sigma Y, Z] \stackrel{\Sigma q^*}{\longleftarrow} \dots \]
la cual se deduce del siguiente resultado:
\begin{teor}
Para todo espacio $Z$, la sucesión 
\[ [X, Z] \stackrel{f^*}{\longleftarrow} [Y, Z] \stackrel{q^*}{\longleftarrow} [C_f, Z] \]
inducida por $X \stackrel{f}{\longrightarrow} Y \stackrel{q}{\longrightarrow} C_f$ es exacta.
\end{teor}
\begin{demo}
\footnote{\nts{Volver a mirar demostración}} Tenemos que $ q^* \circ f^* = (q \circ f)^* = (c_{x_0})^* = c$. Supongamos que $f^*(g) \simeq c$ con $g : Y \longrightarrow Z$, esto es 
\begin{tikzcd}
X \rar{f} \arrow[bend right]{rr}{*} & Y \rar{g} & Z
\end{tikzcd}
pero sabemos que toda aplicación homótopa a la constante se extiende al cono de $X$.
\begin{tabular}{ll}
\begin{minipage}{0.4\textwidth}
\[
\begin{tikzcd}
{} & CX \arrow[bend left]{rd}{h} &  \\
X \rar{f} \arrow[hook]{ur}  & Y \rar{g}  & Z
\end{tikzcd}
\]
\textbf{Añadir imagen p 47}
\end{minipage}
&
\begin{minipage}{0.55\textwidth}
Definimos entonces $k : C_f \longrightarrow Z$ tal que $k\vert_Y = g$, $k \vert_{CX} = h$ \par
Obviamente $q^*(k) = h \circ q = k \circ q' \circ j = g$
\end{minipage}
\end{tabular}
\end{demo}

\section{Fibraciones}
\begin{defin}
Una aplicación $p : E \longrightarrow B$ tiene la propiedad de levantamiento homotópico (HLP) con respecto a un espacio $X$ si dada una homotopía $G : X \times I \longrightarrow B$ que tiene levantamiento en su inicio, entonces se levanta toda ella, esto es, si existe $g : X \longrightarrow E$ tal que  $G(x, 0) = pg(x)$, existe entonces $\widetilde{G} : X \times I \longrightarrow E$ tal que $\widetilde{G}(x,0) = g(x)$ y $G = p \circ \widetilde{G}$. \par
Dicho de otra forma, el diagrama conmutativo siguiente puede completarse con la aplicación punteada $\widetilde{G}$:
\[
\begin{tikzcd}
X \arrow{rr}{g} \arrow[hook]{dd}{i_o} &  & E \arrow{dd}{p} \\
\\
X \times I \arrow{rr}{G} \arrow[dashed]{uurr}{\widetilde{G}} & & B
\end{tikzcd}
\]
\end{defin}
\begin{defin}
Decimos que la aplicación $p : E \longrightarrow B$ es una fibración (de Hurewicz)\footnote{Una fibración de Serre es aquella, más debil, que tiene la HLP para los discos $D^n$. Se puede probar que estas también tienen la HLP para los CW-complejos} si posee la HLP con respecto a cualquier espacio.
\end{defin}
A continuación veremos un resultado que nos dice que toda fibración es sobreyectiva, pero existen ejemplos de aplicaciones sobreyectivas que no son fibraciones.
\begin{ejem}
Sea $X$ el llamado ``seno del topólogo'', es decir, la adherencia del grafo de la función $\sen{\frac{1}{x}}$ con $x in (0,1]$.. Esto es:
\[ X = \{0\} \times [-1,1] \cup \{ (x, \sen{\faktor{1}{x}}) : \ 0 < x \leq 1 \} \]
\nts{Añadir imagen p49}
y tomamos $B = [0,1]$ junto con la aplicación $p : X \longrightarrow B$ la proyección sobre la primera componente. \par
Es claro que la aplicación $p$ es sobreyectiva y la identidad $B \longrightarrow B$ no puede levantarse a $x$ ya que no existe una extensión continua de $\sen{\frac{1}{x}}$ a $0$. No obstante, $Id_B \simeq *$ y la constante puede levantarse.
\end{ejem}
\begin{prop}
Toda fibración con base arcoconexa es sobreyectiva.
\end{prop}
\begin{demo}
Sea $b \in B$ y $b_0 \in B$ en la imagen de $p$, $p(e_0) = b_0$. Sea también $\alpha : I \longrightarrow B$ curva con $\alpha(0) = b_0$, $\alpha(1) = b$. Entonces, por ser p fibración, por la HLP existe $\widetilde{\alpha}$ tal que:
\[
\begin{tikzcd}
\{z\} \arrow{rr}{z \ \longmapsto \ e_0} \arrow[hook]{dd} & & E \arrow{dd}{p} \\
\\
\{z\} \times I \arrow{rr}{\alpha} \arrow[dashed]{uurr}{\widetilde{\alpha}} & & B
\end{tikzcd}
\]
y obviamente $p \widetilde{\alpha}(1) = \alpha(1) = b$.
\end{demo}
Esta misma demostración nos dice que todas las curvas se levantan.
\begin{teor}
Todas las fibras tienen el mismo tipo de homotopía.
\end{teor}
\begin{demo}
Denotemos por $F_b = p^{-1}(b)$ la fibra en $b$. Así, para cada $b_0, b_1 \in B$ y cada curva $\omega : I \longrightarrow B$ curva en $B$ uniendo $b_0$ con $b_1$, definimos una aplicación continua $h_[\omega] : F_{b_0} \longrightarrow F_{b_1}$ de la siguiente forma: \par 
Por la HLP, si consideramos 
\begin{align*}
F_{b_0} \times I &\longrightarrow B \\
(e, t) &\longmapsto \omega(t)
\end{align*}
existe un diagrama: \par
\begin{tabular}{ll}
\begin{minipage}{0.5\textwidth}
\[
\begin{tikzcd}
F_{b_0} \arrow[hook]{rr} \arrow[hook]{dd}{i_0} & & E \arrow{dd}{p} \\
\\
F_{b_0} \times I \arrow{rr} \arrow{uurr}{G} & & B
\end{tikzcd}
\]
\end{minipage}
&
\begin{minipage}{0.5\textwidth}
tal que $p \, G(e,t) = \omega(t)$. \textbf{¿Insertar img p 51?}
\end{minipage}
\end{tabular}
Obviamente, para cada $e \in F_{b_0}$, $G(e,-)$ es una curva en $E$ de forma que $G(e,t) \in F_{\omega(t)} \forall t$. Definimos entonces
\begin{align*}
h_{[w]} : F_{b_0} &\longrightarrow F_{[b_1]} \\
e &\longmapsto G(e,1)
\end{align*}
Es más, si $\omega \simeq_{\{0,1\}} \omega'$ entonces $h_{[\omega]} \simeq h_{[\omega']}$. \par
Tomemos ahora $\omega$ y $\tau$ curvas en $B$ que unen a $b_0$ con $b_1$ y $b_1$ con $b_2$ respectivamente. Sean $F : F_{b_0} \times I \longrightarrow E$, $G: F_{b_1} \times I \longrightarrow E$ tales que $p \, F(e,t) = \omega(t)$ y $p \, G(e,t) = \tau(t)$. Definimos
\begin{align*}
& H : F_{b_0} \times I \longrightarrow E \\
H(e,t) &= \begin{cases}
F(e, 2t) & \text{si } t \leq \frac{1}{2} \\
G(h_{[\omega]}(e), 2t -1) & \text{si } t \geq \frac{1}{2}
\end{cases}
\end{align*}
Se tiene entonces $p \, H(e,t) = (\omega \cdotp \tau)(t)$, por lo que $h_{[\omega \cdotp \tau]} (e) = H(e,1) = G(h_{[\omega]}(e), 1) = 
h_{[\tau]} \circ h_{[\omega]}(e)$, esto es:
\[ h_{[\omega \cdotp \tau]} = h_{[\tau]} \circ h_{[\omega]} \]
Por tanto, $h_{[\omega]} \circ h_{[\omega^{-1}]} = h_{[\omega^{-1} \cdotp \omega]} = h_{[c_b]} = Id_{F_b}$.

\begin{ejems}
\begin{enumerate}
\item La proyección $X \times Y \longrightarrow X$ es una fibración de forma elemental, llamada fibración trivial. Su fibra es $Y$.
\item Hurewicz probó que si, dada una aplicación continua $p: E \longrightarrow B$, existe un recubrimiento $\mathcal{U} = \{U_i\}$ de $B$ tal que $p \vert_{p^{-1}(U_i)} : p^{-1}(U_i) \longrightarrow U_i$ es una fibración para todo $i$ y $B$ es paracompacto, entonces $p$ es una fibración. \par
Con esto, todo fibrado con base paracompacta es una fibración. Para verlo, basta con aplicar el resultado de Hurewicz a una trivialización del fibrado $\{ U_\alpha \}_{\alpha \in I}$ recubrimiento de $B$, donde para cada $\alpha$ existe $\varphi_\alpha : U_\alpha \times F \longrightarrow p^{-1}(U_\alpha)$ tal que
\[
\begin{tikzcd}
U_\alpha \times F \arrow{rr}{\cong} \arrow{dr}{p_1} & & p^{-1}(U_\alpha) \arrow{ld}{p} \\
& U_\alpha &
\end{tikzcd}
\]

\item Todo espacio recubridor es una fibración de fibra discreta.
\end{enumerate}
\end{ejems}
\end{demo}

\subsection*{Fibraciones inducidas o pullbacks}
Dada $p : E \longrightarrow B$ una fibración y $f : X \longrightarrow B$ una aplicación . Formamos el siguiente diagrama conmutativo
\[
\begin{tikzcd}
f^*(E) \arrow{rr}{p_E} \arrow{dd}{p_X} & & E \arrow{dd}{p} \\
\\
X \arrow{rr}{f} & & B
\end{tikzcd}
\]
$f^*(E) = \{ (x,e) \in X \times E : f(x) = p(e) \}$, esto es, $f^*(E)$ es el verdadero pullback de $p$ y $f$ Pues bien, se tiene entonces:
\begin{teor}
La aplicación 
$\begin{aligned}
f^*(E) &\stackrel{p_X}{\longrightarrow} X \\
(x,e) &\longmapsto x
\end{aligned}$ es una fibración.
\end{teor}
\begin{demo}
Por ser $p$ fibración, existe $\widetilde{H}$ como en el diagrama
\[
\begin{tikzcd}
Z\rar{g} \dar[hook]{i_0} & f^*(E) \rar{p_E} \dar{p_X} & E \dar{p} \\
Z\times I \rar{H} \arrow[dashed, bend left=5]{ur}{G} \arrow[dashed, bend right=5]{rru}{\widetilde{H}} & X \rar{f} & B
\end{tikzcd}
\]
Definimos entonces $G(z,t) = (H(z,t), \widetilde{H}(z,t))$. Veamos que se verifica todo:
\begin{enumerate}
\item $G(z,t) \in f^*(E)$ ya que $f \, H(z,t) = p \, \widetilde{H}(z,t)$.
\item Claramente $p_X \, G = H$.
\item $G \circ i_0 = g$. Sabemos que $\widetilde{H} \, i_0 = p_e \, g$ por lo que la segunda coordenada de $g$ es la segunda coordenada de $G \circ i_0$ (que es $\widetilde{H}$) y la primera coordenada de $g$ es $H(z,0)$ que es la primera de $G \circ i_0$. 
\end{enumerate}
\end{demo}
Si consideramos ahora $A \subset B$ tenemos el siguiente resultado:
\begin{coro}
Si $p : E \longrightarrow B$ es fibración, entonces $p : p^{-1}(A) \longrightarrow A$ también lo es.
\end{coro}
\begin{prop}
En un pullback las fibras coinciden.
\end{prop}
\begin{tabular}{ll}
\begin{minipage}{0.3\textwidth}
\[
\begin{tikzcd}
F \rar{Id} \dar & F \dar \\
f^*(E) \rar \dar{p_X}  & E \dar{p} \\
X \rar & B
\end{tikzcd}
\]
\end{minipage}
&
\begin{minipage}{0.65\textwidth}
\begin{demo}
Sea $b_0$ el punto base de $B$. $F = \nobreak p^{-1}(b_0)$. Por otra parte, si $x_0$ es el punto base de $X$, $p_X^{-1}(x_0) = \nobreak \{ (x_0, e) : p(e) = f(x_0) = b_0\} \cong F_{b_0}$
\end{demo}
\end{minipage}
\end{tabular}
Vamos ahora a ver un caso particular importante. Primero necesitaremos el siguiente resultado:
\begin{prop}
La aplicación $X^I \stackrel{\gamma}{\longrightarrow} X$, $\gamma(\omega) = \omega(1)$ es una fibración.
\end{prop}
\begin{demo}
Consideramos el siguiente diagrama:\par
\begin{tabular}{ll}
\begin{minipage}{0.3\textwidth}
\[
\begin{tikzcd}
Z \arrow{rr}{g} \arrow[hook]{dd}{i_0} & & X^I \arrow{dd}{\gamma} \\
\\
Z \times I \arrow{rr}{H} \arrow[dashed]{uurr}{\widetilde{H}} & & X
\end{tikzcd}
\]
\end{minipage}
&
\begin{minipage}{0.65\textwidth}
Definimos $\widetilde{H}$ como sigue:
\[ \widetilde{H}(z,t) = g(z) \cdotp H(z, [0,t]) \]
esto es, como la composición de la curva $g(z)$ con la curva $H_z$ de $0$ a $t$. Está bien definido ya que
\begin{align*}
g(z)(1) = \gamma(g(z)) = H(z,0)\\
\widetilde{H}(z,0) = g(z) \\
\widetilde{H}(z,t)(1) = \gamma \circ \widetilde{H}(z,t) = H(z,t) \end{align*}
\end{minipage}
\end{tabular}
De forma explícita:
\[
\widetilde{H}(z,t)(s) = 
\begin{cases}
g(z)(s(t+1)) & \text{si } 0 \leq s \leq \frac{1}{t+1} \\
H(z, s(t+1) -1) & \text{si } \frac{1}{t+1} \leq s \leq 1
\end{cases}
\]
\end{demo}
Por el corolario anterior, también lo es por tanto la llamada ``Fibración de caminos'' \par
\begin{tabular}{ll}
\begin{minipage}{0.5\textwidth}
\nts{Insertar imagen p56}
\end{minipage}
&
\begin{minipage}{0.5\textwidth}
\begin{align*}
PX \longrightarrow X \\
\omega \longmapsto \omega(1)
\end{align*}
\[PX = \left\lbrace \omega \in X^I : \omega(0) = x_0 \right\rbrace \]
cuya fibra es $\Omega X$.
\end{minipage}
\end{tabular}
\[ \Omega X \longrightarrow PX \longrightarrow X \]
Por otra parte, dada $f: X \longrightarrow Y$ una aplicación cualquiera, podemos descomponerla de la siguiente forma
\[
\begin{tikzcd}
X \arrow{rr}{f} \drar{\varphi} & & Y \\
 & E_f = \{ (x, \omega ) \in X \times Y^I : f(x) = \omega(0) \} \urar[twoheadrightarrow]{p_1} &
\end{tikzcd}
\]
Donde $E_f$ es el pullback de $p$ y $f$ y $\varphi(x) = (x, C_{f(x)})$ y $p_1(x, \omega) = \omega(1)$. \par
Tenemos que $p_1$ es la composición de una proyección de un producto (que es fibración) y de evaluar en $1$ que acabamos de ver que también lo es. Luego $p_1$ es una fibración. Por otra parte, $\varphi$ es también de forma obvia una equivalencia de homotopía. Por lo tanto, tenemos el siguiente resultado:
\begin{teor}
Toda aplicación es, salvo homotopía, una fibración. \footnote{\nts{Mirar en libro}}
\[
\begin{tikzcd}
X \arrow{rr}{f} \drar{\varphi} & & Y \\
 & E_f  \urar[twoheadrightarrow]{p_1} &
\end{tikzcd}
\]
\end{teor}
A la fibra de $p_1$ se le denomina fibra homotópica de $f$ y es:
\begin{align*}
F &\longrightarrow E_f \stackrel{p_1}{\longrightarrow} Y \\
F = \{ (x, \omega) & : \omega(1) = y_0, \omega(0) = f(x) \}
\end{align*}
Si consideramos la inversa homotópica de $\varphi$, $E_f \stackrel{\phi}{\longrightarrow} X$, $\phi(x, \omega) = x$, entonces la cmposición  \nts{corregir}
\[
\begin{tikzcd}
F \rar & E_f \rar{\phi} & X \\
(x, \omega) \arrow[mapsto]{rr} & & x
\end{tikzcd}
\]
es fibración cuya fibra típica es $\{(x_0, \omega) : \omega(0) = f(x_0) = y_0, \ \omega(1) = y_0 \} = \Omega Y$. \par
Esto da lugar a la sucesión
\[ \Omega Y \longrightarrow F \longrightarrow X \stackrel{f}{\longrightarrow} Y \]
Si repetimos este proceso una y otra vez, obtenemos:
\[ \dots \longrightarrow \Omega^2 Y \longrightarrow \Omega F \longrightarrow \Omega X \longrightarrow \Omega Y \longrightarrow F \longrightarrow X \longrightarrow Y \]
Que es dual en el sentido de Eckmann-Hilton de la sucesión de Barratt-Puppe. Aplicando $[Z,-]$: 
\[ \dots  \longrightarrow [Z,\Omega F] \longrightarrow [Z, \Omega X] \longrightarrow [Z,\Omega Y] \longrightarrow [Z, F] \longrightarrow [Z, X] \longrightarrow [Z, Y] \]
Se puede demostrar que:
\begin{teor}
La sucesión
\[ \dots  \longrightarrow [Z,\Omega F] \longrightarrow [Z, \Omega X] \longrightarrow [Z,\Omega Y] \longrightarrow [Z, F] \longrightarrow [Z, X] \longrightarrow [Z, Y] \]
es una sucesión exacta de grupos (o de acción o conjuntos en los casos pertinentes).
\end{teor}
Entonces, tomando en el teorema $Z = S^0$, obtenemos la sucesión
\[ \dots  \longrightarrow [S^0,\Omega^n F] \longrightarrow [S^0, \Omega^n X] \longrightarrow [S^0,\Omega^n Y] \longrightarrow [S^0, \Omega^{n-1} F] \longrightarrow [S^0, \Omega^{n-1} X] \longrightarrow \dots \]
Pero ya vimos que $[\Sigma X, Y] \cong [X, \Omega Y]$, luego tenemos la sucesión exacta:
\[ \dots  \longrightarrow [\Sigma^n S^0, F] \longrightarrow [\Sigma^n S^0,  X] \longrightarrow [\Sigma^n S^0, Y] \longrightarrow [\Sigma^{n-1} S^0, F] \longrightarrow [\Sigma^{n-1}S^0,  X] \longrightarrow \dots \]
Y tenemos que $\Sigma^n S^0 = S^n$ y $[S^n, X] = \pi_n(X)$. Por lo tanto, tenemos el siguiente resultado:
\begin{teor}
Se tiene la siguiente sucesión exacta larga en homotopía de una fibración:
\[
\dots \longrightarrow \pi_n (F) \longrightarrow \pi_n (E) \longrightarrow \pi_n (B) \longrightarrow \pi_{n-1} (F) \longrightarrow \dots 
\]
\end{teor}