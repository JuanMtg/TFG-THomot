\chapter{Fibraciones y Cofibraciones}
En esta parte introduciremos los conceptos de fibraciones y cofibraciones, relacionándolas con lo anteriormente visto.
\section[sdf]{Cofibraciones}
\begin{defin}
Sea $A \subset X$ un subespacio. Decimos que $i : A \longhookrightarrow X$ o el par $(X, A)$ tiene la propiedad de extensión homotópica (HEP) con respecto al espacio $Y$ si dada una homotopía
\[ G : A \times I \longrightarrow Y \]
que tiene una extensión a X en su inicio, esto es, si existe $f: X \longrightarrow Y $ tal que $f(a) = G(a, 0)$ $\forall \ a \in A$, entonces tienen una extensión a toda ella a $X \times I$, es decir, existe $F : X \times I \longrightarrow Y$ tal que $f(x) = F(x,0)$ y $F(a,t) = G(a, t)$ $\forall \ t \in I, \ a \in A$ \par 
De forma equivalente, $(X,A)$ tiene la HEP si el siguiente diagrama puede completarse con el morfismo punteado:
\[
\begin{tikzcd}
	{} 						  & X \drar{i_0} \arrow[bend left]{drr}{f}			   &        					&   \\
	A \urar[hook] \drar{i_0}  &   												   & X \times I \rar[dashed]{F} & Y \\
	   						  & A \times I  \urar[hook] \arrow[bend right]{urr}{G} &   							&
\end{tikzcd}
\]
\end{defin}
\begin{defin}
Decimos que un par $(X,A)$ ( o la inclusión $i: A \longhookrightarrow X$) es una cofibración si posee la HEP con respecto a todo espacio Y. \par 
Es decir, $A \subset X$ es cofibración si toda aplicación  
\[ f : X \times \{0\} \cup A \times I \longrightarrow Y \]
se extiende a $X \times I$.

\end{defin}