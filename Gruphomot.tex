\chapter{Grupos de Homotopía}
\section{Grupos de homotopía de mayor orden}
El grupo fundamental $\homot{1}{X, x_0}$ ya es conocido. Además, anteriormente ya hemos definido los grupos de homotopía de un espacio:
\[ \homot{n}{X, x_0} = [(I^n, \partial I^n), (X, x_0)], \text{ donde } I^n = [0,1]^n \]
Definimos, dadas $f,g : (I^n, \partial I^n) \longrightarrow (X, x_0)$,
\[
(f + g)(s_1, \ldots, s_n) = 
\begin{cases}
f(2s_1, s_2, \ldots, s_n) & \text{si } s_1 \leq \frac{1}{2} \\
g(2s_1 - 1, s_2, \ldots, s_n) & \text{si } s_1 \geq \frac{1}{2}
\end{cases}
\]
Esta operación induce en las clases de homotopía una operación que es asociativa:
\[
(f + g) + h \simeq f + (g + h) \qquad (\text{relativa a } \partial I^n)
\]
admite elemento neutro:
\[
f + c_{x_0} \simeq f \simeq c_{x_0} \qquad (\text{relativa a } \partial I^n)
\]
y elemento inverso:
\[
f^{-1} (s_1, \ldots, s_n) = f(1 - s_1, \ldots, s_n) \qquad (\text{relativa a } \partial I^n)
\]
Se podría pensar sobre otras posibles operaciones cambiando la coordenada elegida. Podemos definir las operaciones
\[
(f +_i g)(s_1, \ldots, s_n) = 
\begin{cases}
f(s_1, \ldots, 2s_i, \ldots, s_n) & \text{si } s_i \leq \frac{1}{2} \\
g(s_1, \ldots, 2s_i -1, \ldots, s_n) & \text{si } s_i \geq \frac{1}{2} \\
\end{cases}
\]
Pero resulta que estas operaciones coinciden. Esto puede verse haciendo uso del siguiente resultado:
\begin{prop}[(Argumento de Eckmann-Hilton)]
Sea $X$ un conjunto dotado de dos operaciones $\bullet$ y $\circ$ y supongamos que 
\begin{enumerate}
\item $\bullet$ y $\circ$ poseen la misma unidad.
\item $\forall a,b,c,d$ se verifica que
\[
(a \bullet b) \circ (c \bullet d) = (a \circ c) \bullet (b \circ d) 
\]
\end{enumerate}
Entonces $\bullet$ y $\circ$ coinciden y son asociativas y conmutativas.
\end{prop}
\begin{demo}
Sean $a,b \in X$ y sea $1$ la unidad de ambas operaciones. Entonces
\begin{align*}
a \circ b &= (1 \bullet a) \circ (b \bullet 1) = (1 \circ b) \bullet (a \circ 1) = b \bullet a  \\
&= (b \circ 1) \bullet (1 \circ a) = (b \bullet 1) \circ (1 \bullet a) = b \circ a
\end{align*}
\end{demo}
Por tanto, como estas operaciones verifican las condiciones de la proposición, obtenemos el siguiente resultado inmediatamente:
\begin{teor}
Si $n \geq 2$, entonces $\homot{n}{X, x_0}$ es abeliano.
\end{teor}
El caso $n=1$ no puede contemplarse ya que al trabajar únicamente con una coordenada no disponemos de otra operacioón distinta con l aque aplicar el Argumento de Eckmann-Hilton.
\begin{custom}[Observaciones]
\begin{enumerate}
\item En el caso que tengamos $C$ componente arcoconexa que contiene a $x_0$, entonces $\homot{n}{X, x_0} = \homot{n}{C, x_0}$. Por lo tanto, supondremos $X$ arcoconexo.

\item Las aplicaciones $ (I^n, \partial I^n) \longrightarrow (X, x_0)$ coinciden con las aplicaciones $(\faktor{I^n}{\partial I^n}, \ast) \longrightarrow (X, x_0)$. Pero $S^n = \faktor{I^n}{\partial I^n}$ y con estas identificaciones, la estructura de grupo es la que conocíamos \par
\nts{Insertar imagen p63}
\[S^n \longrightarrow S^n \vee S^n \stackrel{f \vee g}{\longrightarrow} X \]
\end{enumerate}
\end{custom}
Veamos ahora que si cambiamos el punto base obtenemos grupos de homotopía isomorfos (al igual que ocurría para $\pi_1$). \par
Tomamos $2$ puntos $x_0, x_1 \in X$ y un camino $\gamma : I \longrightarrow X$ tal que $\gamma(0) = x_0$, $\gamma(1) = x_1$. A cada aplicación $f : (S^n, p_o) \longrightarrow (X, x_0)$ le asociamos una aplicación
\[
\gamma_f : (S^n, p_0) \longrightarrow (X, x_0)
\]
definida de la siguiente forma: \par
Encogemos el dominio de f a un cubo menor concéntrico en $I^n$, e insertamos el camino $\gamma$ en cada segmento radial entre este cubo y $\partial I^n$. \nts{Insertar imagen p64, explicar $I^n$} \par
Se tienen entonces las siguientes propiedades: \par
\begin{enumerate}
\item $\gamma_{(f+g)} \simeq \gamma_f + \gamma_g$
\item $(\gamma \eta)_f \simeq \gamma_{\eta_f}$
\item ${c_{x_1}}_f \simeq f$
\end{enumerate}
Entonces para cada $\gamma$ definimos
\begin{align*}
\beta_\gamma : \homot{n}{X, x_1} &\longrightarrow \homot{n}{X, x_0} \\
[f] &\longmapsto [\gamma_f]
\end{align*}
que es un morfismo de grupos, cuyo inverso es $\beta_{\gamma^{-1}}$. Por tanto:
\begin{teor}
Dados $x_0, x_1 \in X$, se tiene que $\homot{n}{X, x_1} \cong \homot{n}{X, x_0}$ mediante $\beta_\gamma$.
Es más, si $n=1$,
\begin{align*}
\homot{1}{X, x_1} &\stackrel{\cong}{\longrightarrow} \homot{1}{X, x_0} \\
[\alpha] &\longmapsto [\gamma \alpha \gamma^{-1}]
\end{align*}
\end{teor}
De las propiedades también se obtiene
\begin{teor}
La aplicación 
\begin{align*}
\homot{1}{X, x_0} \times \homot{n}{X, x_0} &\longrightarrow \homot{n}{X, x_0} \\
[\gamma], [f] &\longmapsto [\gamma_f]
\end{align*}
es una acción.
\end{teor}
Cuando $n = 1$, la acción es por conjugación. hace al grupo abeliano $\homot{n}{X, x_0}$ un $\bb{Z}[\homot{1}{X}]$-módulo via
\[
\left( \sum_i n_i \gamma_i \right) \alpha = \sum_i n_i {\gamma_i}_\alpha
\]
Decimos que el espacio es abeliano si la acción de $\homot{1}{X, x_0}$ en $\homot{n}{X, x_0}$ es trivial. En particular, si $n=1$, esta condición nos dice que $\homot{1}{X, x_0}$ es abeliano. \par
Si tenemos $f : (X, x_0) \longrightarrow (Y, y_0)$ aplicación continua, podemos definir
\begin{align*}
\homot{n}{f} : \homot{n}{X, x_0} &\longrightarrow \homot{n}{Y, y_0} \\
[\alpha] & \longmapsto [f \circ \alpha]
\end{align*}
Con esta definición, tenemos:
\begin{prop}
$\pi_n$ es un funtor en la categoría homotópica con valores en la categoría de los grupos:
\begin{enumerate}
\item $\homot{n}{f}(a+b) = \homot{n}{f}(a) + \homot{n}{f}(b)$
\item $\homot{n}{Id} = Id_{\homot{n}{X}}$
\item Si $f \simeq g$, entonces $\homot{n}{f} = \homot{n}{g}$.
\end{enumerate}
\end{prop}
Por tanto, si $f : (X, x_0) \longrightarrow (Y, y_0)$ es una equivalencia de homotopía, entonces $\homot{n}{f}$ es un isomorfismo. En efecto, existe $g$ tal que $f \circ g \simeq Id_{(Y, y_0)}$ y $g \circ f \simeq Id_{(X, x_0)}$. Tenemos que $\homot{n}{f \circ g} = \homot{n}{f} \circ \homot{n}{g}$ ya que:
\[
\homot{n}{f \circ g}([\alpha]) = [f \circ g \circ \alpha] = \homot{n}{f}([g \circ \alpha]) = \homot{n}{f} \circ \homot{n}{g}([\alpha])
\]
Luego tenemos que 
\begin{align*}
\homot{n}{f \circ g} = \homot{n}{f} \circ \homot{n}{g} = Id \\
\homot{n}{g \circ f} = \homot{n}{g} \circ \homot{n}{f} = Id
\end{align*}
Y así, $\homot{n}{f}$ es un isomorfismo.
\section{Grupos de homotopía relativa}
Vamos a generalizar los grupos de homotopía. Para ello construiremos los llamados grupos de homotopía relativa. Para ello tomamos $x_0 \in A \subset X$ y consideramos $I^{n-1} \subset I^n$ como los puntos con última coordenada $0$ \nts{Añadir imagen p68}
y tomamos $J^{n-1} = \partial I^n - I^{n-1}$. \par
Se define entonces el grupo de homotopía relativa $n$-ésimo como
\[
\homot{n}{X, A, x_0} = [(I^n, \partial I^n, J^{n-1}), (X, A, x_0)]
\]
Se verifica que $\homot{n}{X, x_0} = \homot{n}{X, x_0, x_0}$. \par
Con la anterior operación definida, $\homot{n}{X, A, x_0}$ es un grupo si $n \geq 2$ y es abeliano si $n \geq 3$. \par
En el caso $n = 1$, tenemos que $\partial I = \{0,1\}$, $I^{n-1} = I^0 = \{0\}$, $J^0 = \{0\}$. Por tanto
\[
\homot{1}{X, A, x_0} = \text{clases de homotopía relativa a } x_0 \text{ de caminos qur van a } A \text{ en } x_0.
\]
que no es un grupo. \par
También podemos ver $\homot{n}{X, A, x_0}$ como $[(D^n, S^{n-1}, p_0), (X, A, x_0)]$, ya que colapsando $J^{n-1}$ a un punto, transforma $(I^n, \partial I^n, J^{n-1})$ en $(D^n, S^{n-1}, p_0)$ y le damos la estructura de grupo \nts{Insertar imagen p70} \par
Tenemos entonces el siguiente resultado:
\begin{teor}
La aplicación $ f : (D^n, S^{n-1}, p_0) \longrightarrow (X, A, x_0)$ representa el elemento neutro de $\homot{n}{X, A, x_0}$ si y sólo si $f \simeq g$ relativa a $S^{n-1}$ para algún $g$ con $Im\ g \subset A$.
\end{teor}
\begin{demo}
Supongamos que $f \simeq g$ relativo a $S^{n-1}$ e $Im \ g \subset A$. Entonces existe $H : (D^n, S^{n-1}, p_0) \longrightarrow (X, A, x_0)$. Por tanto, $[f] = [g]$ en $\homot{n}{X, A, x_0}$. Si tomamos $F: (D^n, p_0) \times I \longrightarrow (D^n, p_0)$ un retracto de deformación de $D^n$ en $p_0$, entonces 
\begin{align*}
g \circ F : D^n \times I \longrightarrow X \\
g \circ F (p,0) = g(p) \in A \\
g \circ F (p,1) = g(p_0) = x_0 \\
g \circ F (p_0, t) = x_0 \ \forall t \in I \\
\end{align*}
Luego $g \simeq 0$ y así $[f]=[g]=0$. \par
Recíprocamente, supongamos que $[f]=0$. Entonces existe $H: (D^n, S^{n-1}, p_0) \times I \longrightarrow (X, A, x_0) $. Si restringimos $H$ a una familia de discos $D^n \times I$ empezando por $D^n \times \{0\}$ y terminando por $D^n \times \{1\} \cup S^n \times I$, tenemos $F_t: H \vert_{D^n \times \{t\} \cup S^{n-1} \times [0, t]}$ \par
Todos los $n$-discos tienen el mismo borde $S^{n-1}$, luego $F : f \simeq g$ relativo a $S^{n-1}$ e $Im \ g \subset A$. \par
\nts{Mirar de nuevo, buscar en Spanier P372}
\end{demo}

Al igual que hicimos para los grupos de homotopía, dada una aplicación $\varphi :  (X, A, x_0) \longrightarrow (Y, B, y_0)$, esta induce un morfismo entre los grupos de homotopía relativa, $\homot{n}{\varphi} : \homot{n}{X, A, x_0} \longrightarrow \homot{n}{Y,B,y_0}$. Por tanto, considerando las inclusiones $i : (A, x_0) \longrightarrow (X, x_0)$ y $j: (X, x_0, x_0) \longrightarrow (X, A, x_0)$, éstas determinan morfismos
\begin{align*}
\homot{n}{i}&: \homot{n}{A, x_0} \longrightarrow \homot{n}{X, x_0} \\
\homot{n}{j}&: \homot{n}{X, x_0, x_0} \longrightarrow \homot{n}{X, A, x_0}
\end{align*}
Por otra parte, dada una aplicación $f: (D^n, S^{n-1}, p_0) \longrightarrow (X, A, x_0)$, la restricción $f \vert_{S^{n-1}} : (S^{n-1}, p_0) \longrightarrow (A, x_0)$ determina un morfismo $\partial : \homot{n}{X, A, x_0} \longrightarrow \homot{n-1}{A, x_0}$ llamado aplicación borde. \par
Entonces, tenemos el siguiente resultado:
\begin{teor}
La sucesión
\begin{align*} 
\ldots \longrightarrow \homot{n}{A, x_0} \stackrel{\homot{n}{i}}{\longrightarrow} \homot{n}{X, x_0} \stackrel{\homot{n}{j}}{\longrightarrow} \homot{n}{X, A, x_0} \stackrel{\partial}{\longrightarrow} \homot{n-1}{A, x_0} \longrightarrow \ldots \\
\ldots \longrightarrow \homot{1}{X, A, x_0} \longrightarrow \homot{0}{A, x_0} \homot{0}{X, x_0}
\end{align*}
es exacta.
\end{teor}
\begin{demo}
Veamos la exactitud en $\homot{n}{X, x_0}$. Consideremos la composición $\homot{n}{i} \circ \homot{n}{j} = \homot{n}{k}$. Veamos que es $0$. $k = i \circ j : (A, x_0, x_0) \longrightarrow (X, A, x_0)$. Consideremos ahora una función $f : (D^n, S^{n-1}, p_0) \longrightarrow (A, x_0, x_0)$. Así $k \circ f : (D^n, S^{n-1}, p_0) \longrightarrow (X, A, x_0)$ representa al neutro de $\homot{n}{X, A, x_0}$ por el teorema anterior. \par
Veamos ahora que $Ker \ \homot{n}{j} \subset Im \ \homot{n}{i}$. Sea $[f] \in Ker \ \homot{n}{j}$. Entonces $[f]$ representa al neutro en $\homot{n}{X, x_0}$. Por tanto, existirá $g$ tal que $f \simeq g$ relativa a $S^{n-1}$ con $Im \ g \subset A$. Así, $\homot{n}{i}[g] = [f] \in Im \ \homot{n}{i}$. \par
\nts{¿Completar con el resto de demost de exactitud?}
\end{demo}

\section{$n$-conexidad de un espacio}
\begin{defin}
Un espacio topológico $X$ con punto base $x_0$ se dice $n$-conexo si $\homot{i}{X, x_0} = 0$ para todo $i \leq n$.
\end{defin}
Con esta definición, que un espacio sea $0$-conexo quiere decir que es arcoconexo, y que sea $1$-conexo significa que el espacio es símplemente conexo. \par
Como $n$-conexidad implica $0$-conexidad, la mención del punto base de un espacio $n$-conexo no es importante. \par
Así, las siguientes condiciones son equivalentes:
\begin{enumerate}
\item $\homot{n}{X, x_0} = 0$ para todo $i \leq n$ y todo $x_0 \in A$
\item Cada aplicación $S^i \longrightarrow X$ es homotópicamente trivial para todo $i \leq n$.
\item Cada aplicación $S^i \longrightarrow X$ se extiende a $D^{n+1}$ para todo $i \leq n$.
\end{enumerate}
De la misma forma, para un par $(X, A)$ son equivalentes las siguientes condiciones:
\begin{enumerate}
\item Cada aplicación $(D^i, S^{i-1}) \longrightarrow (X, A)$ es homótopa relativa a $S^{n-1}$ a una aplicación $D^i \longrightarrow A$.
\item Cada aplicación $(D^i, S^{i-1}) \longrightarrow (X, A)$ es homótopa en $[(D^i, S^{i-1}), (X, A)]$ a una aplicación $D^i \longrightarrow A$.
\item Cada aplicación $(D^i, S^{i-1}) \longrightarrow (X, A)$ es homótopa en $[(D^i, S^{i-1}), (X, A)]$ a una constante $D^i \longrightarrow A$.
\item $\homot{n}{X, A, x_0} = 0$ $\forall x_0 \in A$.
\end{enumerate}
Para el caso $i = 0$, $\homot{0}{X, A, x_0}$ no está definido, por tanto, decimos que el par $(X, A)$ es $n$-conexo si se verifica una de las cuatro condiciones para $i > 0$ y si se verifica una de las tres primeras cuando $i=0$.

\subsection*{Aplicaciones de la sucesión exacta de un par $(X,A)$}
Veamos ahora varios resultados que hacen uso de esta sucesión exacta.
\begin{teor}
Sea $p : E \longrightarrow B$ una fibración. Tomamos $b_0 \in B$ y un punto $x_0 \in F = p^{-1}(b_0)$ de la fibra de $b_0$. Entonces
\[
\homot{n}{p} : \homot{n}{E, F, x_0} \longrightarrow \homot{n}{B, b_0}
\]
es isomorfismo si $n \geq 1$.
\end{teor}