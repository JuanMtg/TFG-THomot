\addcontentsline{toc}{chapter}{Introducción}
\chapter*{Introducción}
\section*{Los problemas clásicos de levantamiento: extensión y clasificación}\label{sec:levext}
En distintos contextos matemáticos encontramos dos problemas básicos, común a todos ellos, si los despojamos de las características propias de un entorno: los problemas de extensión y levantamiento. \par

\subsection*{El problema de la extensión}\label{c1:ext}
Dado un diagrama de aplicaciones continuas 
$$
\begin{tikzcd}
	A \rar{f} \dar[hook]{i} & Y \\
	X \urar[dashed]{\tilde{f}}	&
\end{tikzcd} 
$$
donde $i$ es una ``inclusión'' en un contexto dado, ¿cuándo existe $\tilde{f}$, extensión de $f$? \par

La respuesta no es trivial. Es claro que, incluso si $i$ no es inclusión, ha de verificarse que si $i(a)= i(b)$ entonces $f(a) = f(b)$. Pero incluso si $i$ es biyectiva es necesario exigirle a $X$, para que la única $\tilde{f}$ posible sea continua, que posea la topología de la identificación determinada por $i$: 
$$
\theta \subset X \text{ es abierto si y sólo si } i^{-1} (\theta) \text{ es abierto de } A. 
$$
\newpage
Hay varios ejemplos de resultados que dan respuesta a este problema. Algunos dan una respuesta positiva:
\begin{teor} 
Si $X,Y$ son espacios métricos, $Y$ completo, y $A$ es denso en $X$, toda aplicación uniformemente continua $f:A\longrightarrow Y $ se extiende a una uniformemente continua $\tilde{f} : X \rightarrow Y$.
\end{teor} 

\begin{teor}[de Extensión de Tietze]
Sea $X$ normal, $A \subset X$ cerrado, $I \subset \mathbb{R}$ intervalo. Entonces toda $f : A \rightarrow I $  admite una extensión $\tilde{f} : X \rightarrow I$.
\end{teor}
\hypertarget{c1t:retractoh}{Mientras que la de otros es negativa:}
\begin{teor} \label{c1t:retracto}
La esfera $S^{n}$ no es un retracto del disco $D^{n+1}$. En otras palabras, la identidad $S^{n} \rightarrow S^{n}$ no se extiende a $D^{n+1}$.
\end{teor}

\subsection*{El problema del levantamiento}\label{c1:lev}
Dualmente \footnote{Dual en el sentido de ``Eckmann-Hilton'', dualidad que se definirá más adelante.},
nos encontramos con el problema del levantamiento de aplicaciones continuas. Es el siguiente: dado un diagrama
$$
\begin{tikzcd}
	{}	& X \dar[two heads]{p} \\
	Y \urar[dashed]{\tilde{f}} \rar{f} & B
\end{tikzcd}
$$
donde $p$ es sobreyectiva ¿cuándo existe $\tilde{f}$, levantamiento de $f$?\\ 
La respuesta tampoco es elemental. Incluso si $p$ es biyectiva hay que exigirle que sea homeomorfismo para que la única $\tilde{f}$ sea continua.\par

Igualmente hay resultados clásicos en distintos ambientes que contestan parcialmente esta cuestión.

\begin{teor} 
Sea $A \subset \mathbb{R}^2$ un conjunto estrellado respecto a $x_{0} \in A$ y $f : A \longrightarrow S^{1}$ una aplicación continua. Entonces f queda determinada de forma continua por su función angular, esto es, existe una aplicación $\tilde{f} : A \longrightarrow \mathbb{R}$ de tal forma que el siguiente diagrama es conmutativo:
$$
\begin{tikzcd}
	{}	& \mathbb{R} \dar{exp} \\
	A \urar{\tilde{f}} \rar{f} & S^1
\end{tikzcd}
$$
Esto es, $f(x) = (cos\tilde{f}(x), sen\tilde{f}(x))$. Es más, dos tales funciones $\tilde{f}$ se diferencian en un múltiplo entero de $2\pi$.
\end{teor}

Estos problemas son, en parte, origen de la teoria de homotopía. Muchas veces, los problemas de extensión y levantamiento son puramente homotópicos: \par
\begin{quotation}
Hay aplicaciones $p : X \longrightarrow B$ para las que existe el levantamiento de $f : \nobreak Y \longrightarrow B$ si existe el levantamiento de una ``deformada'' de $f$. \\
De igual forma existen algunas aplicaciones $i : A \longrightarrow X$ para las que existe una extensión de $g : A \longrightarrow Y$ si y sólo si existe alguna extensión para una ``deformada'' de g.
\end{quotation}
Introduzcamos el concepto de deformación en homotopía:
\begin{defin}
Dos aplicaciones $f, g : X \longrightarrow Y$ se dicen homótopas (deformables la una en la otra), denotado por $f \simeq g$, si existe  $H : X \times [0,1] \longrightarrow Y$ una aplicación tal que $H(x ,0) = f(x)$ y $H(x, 1) = g(x)$.
\end{defin}
Llegados a este punto, los problemas de extensión y levantamiento deparan ahora a un problema común de clasificación. ¿Cuándo dos aplicaciones $X \longrightarrow Y$ son homótopas? ¿Cómo es el conjunto $[X, Y]$ de clases de homotopía de aplicaciones $X \longrightarrow Y$? \par 
Los métodos que se siguen son, a grosso modo, de dos enfoques distintos:\\
Por una parte, a los espacios se le asocian modelos algebraicos y morfismos entre los respectivos modelos algebraicos a las aplicaciones. Estas asociaciones permiten parcialmente su clasificación. \par 
Como ejemplo, probemos el \hyperlink{c1t:retractoh}{teorema anterior} por el que $S^n$ no es un retracto de $D^{n+1}$. Para ello, asociemos a $S^n$ un invariante algebraico no nulo (como por ejemplo el grupo de homología $n$-ésimo). Como el disco $D^{n+1}$ es deformable a un punto, todos esos invariantes se hacen 0. Si $S^n$ fuese retracto de $D^{n+1}$ existiría un diagrama como el siguiente:
$$
\begin{tikzcd}
	S^n \arrow{rr}{Id_{S^n}} \drar{i} & & S^n \\
		&	D^{n+1} \urar{r} & 
\end{tikzcd}
$$
que algebraicamente daría lugar a 
$$
\begin{tikzcd}
	0 \neq F(S^n) \arrow{rr}{Id} \drar & &  F(S^n) \neq 0 \\
		&	0 \urar & 
\end{tikzcd}
$$
lo que resulta contradictorio. \par

Por otra parte, para saber más sobre $[X, Y]$ es habitual tratar de dotar a este conjunto de otras estructuras (grupo, módulo...) que den luz sobre su comportamiento. \par