\addcontentsline{toc}{chapter}{Introducción}
\fancyhead[LO,RE]{\itshape }
\fancyhead[LE,RO]{\itshape  INTRODUCCIÓN }
\chapter*{Introducción}
\section*{Los problemas clásicos de levantamiento, extensión y clasificación}\label{sec:levext}
En distintos contextos matemáticos encontramos dos problemas básicos, común a todos ellos: los problemas de extensión y levantamiento. \par

\subsection*{El problema de la extensión}\label{c1:ext}
Dado un diagrama de aplicaciones continuas 
$$
\begin{tikzcd}
	A \rar{f} \dar[hook, swap]{i} & Y \\
	X \urar[dashed, swap]{\tilde{f}}	&
\end{tikzcd} 
$$
donde $i$ es una inclusión en un contexto dado, ¿cuándo existe $\tilde{f}$ extensión de $f$? \par

La respuesta no es trivial. Para que esto ocurra, aunque $i$ no sea inclusión, ha de verificarse que si $i(a)= i(b)$ entonces $f(a) = f(b)$. Pero incluso si $i$ es biyectiva es necesario exigirle a $X$, para que la única $\tilde{f}$ posible sea continua, que posea la topología de la identificación determinada por $i$: 
$$
\theta \subset X \text{ es abierto si y sólo si } i^{-1} (\theta) \text{ es abierto de } A. 
$$
\newpage
Hay varios ejemplos de resultados que dan respuesta a este problema. Algunos dan una respuesta positiva:
\begin{teorf} 
Si $X,Y$ son espacios métricos, $Y$ completo, y $A$ es denso en $X$, toda aplicación uniformemente continua $f:A\longrightarrow Y $ se extiende a una uniformemente continua $\tilde{f} : X \rightarrow Y$.
\end{teorf} 

\begin{teorf}[de Extensión de Tietze]
Sea $X$ normal, $A \subset X$ cerrado, $I \subset \mathbb{R}$ intervalo. Entonces toda $f : A \rightarrow I $  admite una extensión $\tilde{f} : X \rightarrow I$.
\end{teorf}
\hypertarget{c1t:retractoh}{Mientras que la de otros es negativa:}
\begin{teorf} \label{c1t:retracto}
La esfera $S^{n}$ no es un retracto del disco $D^{n+1}$. En otras palabras, la identidad $S^{n} \rightarrow S^{n}$ no se extiende a $D^{n+1}$.
\end{teorf}


Muchas veces, el problema de la extensión es puramente homotópico: existen algunas aplicaciones $i : A \longrightarrow X$ para las que existe una extensión de $g : A \longrightarrow Y$ si y sólo si existe alguna extensión para una ``deformada'' de g. \par 
Para intentar dar una respuesta la pregunta de la extensión, veremos la \textit{propiedad de extensión homotópica}, introduciremos las cofibraciones y algunos resultados sobre éstas. \par

%Un par $(X, A)$ con $A \subset X$  (que pueden ser visto como $i : A \longhookrightarrow X$) tiene la propiedad de extensión homotópica si dada una homotopía $A \times I \longrightarrow Y$ con una extensión en su inicio a $X$, entonces se extiende para todo $X \times I$. \par
%
%Con esta propiedad, definiremos una cofibración como los pares $(X, A)$ con  (que también pueden ser vistas como las inclusiones $i : A \longhookrightarrow X$) que verifican la propiedad de extensión homotópica con respecto a cualquier espacio topológico $Y$. \par


\subsection*{El problema del levantamiento}\label{c1:lev}
Dualmente \footnote{Dual en el sentido de ``Eckmann-Hilton'', dualidad de la que se hablará más adelante.},
nos encontramos con el problema del levantamiento de aplicaciones continuas. Es el siguiente: dado un diagrama
$$
\begin{tikzcd}
	{}	& X \dar[two heads]{\ p} \\
	Y \urar[dashed]{\tilde{f}} \rar[swap]{f} & B
\end{tikzcd}
$$
donde $p$ es sobreyectiva, ¿cuándo existe $\tilde{f}$ levantamiento de $f$?\par 
La respuesta tampoco es elemental. Incluso si $p$ es biyectiva hay que exigirle que sea homeomorfismo para que la única $\tilde{f}$ sea continua.\par

Al igual que para el problema de la extensión, para el levantamiento hay resultados clásicos en distintos ambientes que contestan parcialmente esta cuestión.

\begin{teorf} 
Sea $A \subset \mathbb{R}^2$ un conjunto estrellado respecto a $x_{0} \in A$ y $f : A \longrightarrow S^{1}$ una aplicación continua. Entonces f queda determinada de forma continua por su función angular, esto es, existe una aplicación $\tilde{f} : A \longrightarrow \mathbb{R}$ de tal forma que el siguiente diagrama es conmutativo:
$$
\begin{tikzcd}
	{}	& \mathbb{R} \dar{\ exp} \\
	A \urar{\tilde{f}} \rar[swap]{f} & S^1
\end{tikzcd}
$$
Esto es, $f(x) = \left(\text{cos}\tilde{f}(x), \ \text{sen}\tilde{f}(x) \right)$. Es más, dos tales funciones $\tilde{f}$ se diferencian en un múltiplo entero de $2\pi$.
\end{teorf}

Al igual que para el problema de la extensión, el del levantamiento es también homotópico: hay aplicaciones $p : X \longrightarrow B$ para las que existe el levantamiento de $f : \nobreak Y \longrightarrow B$ si existe el levantamiento de una ``deformada'' de $f$. \par

Para el estudio de este problema, haremos uso de la \textit{propiedad del levantamiento homotópico} y de las fibraciones. 

%Diremos que una aplicación $p : E \longrightarrow B$ tiene la propiedad del levantamiento homotópico con respecto a un espacio $X$ si una homotopía $X \times I \longrightarrow B$ que posee un levantamiento en su inicio, entonces se levanta a todo $X \times I$. \par
%
%Y al igual que con las cofibraciones, diremos que una aplicación $p : E \longrightarrow B$ es una fibración si posee la propiedad del levantamiento homotópico con respecto a cualquier espacio $X$.

Estos problemas son, en parte, origen de la teoría de homotopía. Introduzcamos entonces el concepto de deformación en homotopía:
\begin{defin}
Dos aplicaciones $f, g : X \longrightarrow Y$ se dicen homótopas (deformables la una en la otra), denotado por $f \simeq g$, si existe  $H : X \times [0,1] \longrightarrow Y$ una aplicación tal que $H(x ,0) = f(x)$ y $H(x, 1) = g(x)$.
\end{defin}

De esta forma, los problemas de extensión y levantamiento nos llevan ahora a un problema común de clasificación. ¿Cuándo dos aplicaciones $X \longrightarrow Y$ son homótopas? ¿Cómo es el conjunto $[X, Y]$ de clases de homotopía de aplicaciones $X \longrightarrow Y$? \par 
Los métodos que se siguen son de dos enfoques distintos:\par 
Por una parte, a los espacios se le asocian modelos algebraicos y morfismos entre los respectivos modelos algebraicos a las aplicaciones. Estas asociaciones permiten parcialmente su clasificación. \par 
Como ejemplo, probemos el \hyperlink{c1t:retractoh}{teorema anterior} por el que $S^n$ no es un retracto de $D^{n+1}$. Para ello, asociemos a $S^n$ un invariante algebraico no nulo (como por ejemplo el grupo de homología $n$-ésimo). Como el disco $D^{n+1}$ es deformable a un punto, todos esos invariantes se hacen 0. Si $S^n$ fuese retracto de $D^{n+1}$ existiría un diagrama como el siguiente:
$$
\begin{tikzcd}
	S^n \arrow{rr}{Id_{S^n}} \drar{i} & & S^n \\
		&	D^{n+1} \urar{r} & 
\end{tikzcd}
$$
que algebraicamente daría lugar a 
$$
\begin{tikzcd}
	0 \neq F(S^n) \arrow{rr}{Id} \drar & &  F(S^n) \neq 0 \\
		&	0 \urar & 
\end{tikzcd}
$$
lo que resulta contradictorio. \qed \par

Por otra parte, para saber más sobre $[X, Y]$ es habitual tratar de dotar a este conjunto de otras estructuras (grupo, módulo...) que den luz sobre su comportamiento. \par