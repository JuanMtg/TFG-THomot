\chapter{Introducción} 
\section{Los problemas clásicos de levantamiento, extensión y clasificación}
En distintos contextos matemáticos encontramos dos problemas básicos, común a todos ellos, si los despojamos de las características propias de un entorno. \par

Uno de ellos es la extensión: dado un diagrama de aplicaciones continuas 
$$
\begin{tikzcd}
	A \rar{f} \dar[hook]{i} & Y \\
	X \urar[dashed]{\tilde{f}}	&
\end{tikzcd} 
$$
donde $i$ es una ``inclusión'' en un contexto, ¿cuándo existe $\tilde{f}$, extensión de $f$? \par

La respuesta no es trivial. Es claro que, incluso si $i$ no es inclusión, ha de verificarse que si $i(a)= i(b)$ entonces $f(a) = f(b)$. Pero incluso si $i$ es biyectiva es necesario exigirle a $X$, para que la única $\tilde{f}$ posible sea continua, que posea la topología de la identificación determinada (topología cociente) por $i$: 
$$
\theta \subset X \text{ es abierto si y sólo si } i^{-1} (\theta) \text{ es abierto de } A. 
$$

Resultados clásicos en lo que podemos llamar ``topología general'' atribuyen respuestas positivas a este problema:
\begin{teor} 
Si $X,Y$ son espacios métricos, $Y$ completo, y $A$ es denso en $X$, toda aplicación uniformemente continua $f:A\rightarrow Y $ se extiende a una uniformemente continua $\tilde{f} : X \rightarrow Y$.
\end{teor} 

\begin{teor}[de Extensión de Tietze]
Sea $X$ normal, $A \subset X$ cerrado, $I \subset \mathbb{R}$ intervalo. Entonces toda $f : A \rightarrow I $  admite una extensión $\tilde{f} : X \rightarrow I$.
\end{teor}
También hay resultados clásicos, si se quiere de "análisis no lineal" que constituyen respuestas negativas:

\begin{teor} 
La esfera $S^{n}$ no es un retracto del disco $D^{n+1}$. En otras palabras la identidad $S^{n} \rightarrow S^{n}$ no se extiende a $D^{n+1}$.
\end{teor}

Dualmente \footnote{Dual en el sentido de ``Eckmann-Hilton'', dualidad que se definirá más adelante.},
nos encontramos con el problema del levantamiento de aplicaciones continuas: dado un diagrama
$$
\begin{tikzcd}
	{}	& X \dar[two heads]{p} \\
	Y \urar[dashed]{\tilde{f}} \rar{f} & B
\end{tikzcd}
$$
donde $p$ es sobreyectiva ¿cuándo existe $\tilde{f}$, levantamiento de $f$?.\\ 
La respuesta tampoco es elemental. Incluso si $p$ es biyectiva hay que exigirle que sea homeomorfismo para que la única $\tilde{f}$ sea continua.\par

Igualmente hay resultados clásicos en distintos ambientes que contestan parcialmente esta cuestión.

\begin{teor} 
Sea $A \subset \mathbb{R}^2$ un conjunto estrellado respecto a $x_{0} \in A$ y $f : A \rightarrow S^{1}$ una aplicación continua. Entonces f queda determinada de forma continua por su función angular, esto es, existe una aplicación $\tilde{f} : A \rightarrow \mathbb{R}$ de tal forma que el siguiente diagrama es conmutativo:
$$
\begin{tikzcd}
	{}	& \mathbb{R} \dar{exp} \\
	A \urar{\tilde{f}} \rar{f} & S^1
\end{tikzcd}
$$
Esto es, $f(x) = (cos\tilde{f}(x), sen\tilde{f}(x))$. Es más, dos tales funciones $\tilde{f}$ se diferencian en un múltiplo entero de $2\pi$.
\end{teor}

Estos problemas son, en parte, origen de la teoria de homotopía. Muchas veces, los problemas de extensión y levantamiento, son puramente homotópicos: \par

Hay aplicaciones $p : X \rightarrow B$ para las que existe el levantamiento de $f : Y \rightarrow B$ si existe el levantamiento de una ``deformada'' de $f$. \\

De igual forma existen algunas aplicaciones $i : A \rightarrow X$ para las que existe una extensión de $g : A \rightarrow Y$ si y sólo si existe alguna extensión para una ``deformada'' de g. \par

Introduzcamos el concepto de deformación en homotopía:
\begin{defin}
Dos aplicaciones $f, g : X \rightarrow Y$ se dicen homótopas (deformables la una en la otra) si existe $H : X \times [0,1] \rightarrow Y$ tal que $H(x ,0) = f(x)$ y $H(x, 1) = g(x)$.
\end{defin}
Llegados a este punto, los problemas de extensión y levantamiento deparan ahora a un problema común de clasificación. ¿Cuándo dos aplicaciones $X \rightarrow Y$ son homótopas? ¿Cómo es el conjunto $[X, Y]$ de clases de homotopía de aplicaciones $X \rightarrow Y$?\\
Respuesta: los métodos que se siguen son, a grosso modo, de dos enfoques distintos:\\
Por una parte, se le asocian a los espacios modelos algebraicos y a las (clases de homotopía de) aplicaciones, morfismos entre los respectivos modelos algebraicos que permiten parcialmente su clasificación:
Como ejemplo, probemos el teorema anterior por el que $S^n$ no es un retracto de $D^{n+1}$. Para ello, asociemos a $S^n$ alguno de los "cienes" y "cienes" invariantes algebraicos "no nulos". Como quiera que $D^{n+1}$ es deformable a un punto, todos esos invariantes se hacen 0 para el disco. Si $S^n$ fuese retracto de $D^{n+1}$ existiría un diagrama como: \textbf{(insertar diagrama 1 p6)}
que algebraicamente daría lugar a (insertar diagrama 2 p6)lo que resulta contradictorio.
Por otra parte, para saber más sobre $[X, Y]$ es habitual tratar de dotar a este conjunto de otras estructuras (grupo, módulo, ...) que den luz sobre un comportamiento.\\
Para finalizar, nótese que "ya está dicho todo". Hemos hablado de "fibraciones", "cofibraciones", "tipo de homotopía", "invariantes del mismo" y "dualidad de "Eckman-Hilton"".
