\chapter{Resultados (por añadir)}
\section{Dualidad de Eckmann-Hilton}
Este principio de dualidad, en su forma más básica, consiste en la idea de, dado un diagrama para una construcción, invertir el sentido de las flechas de dicho diagrama. \par 
Un ejemplo de esto son las fibraciones y cofibraciones. Estas dos construcciones son duales en el sentido de Eckmann-Hilton. \par
Una fibración $p : E \longrightarrow B$ verifica que posee la HLP, que se representa por el diagrama:
\[
\begin{tikzcd}
X \arrow{rr}{g} \arrow[hook]{dd}{i_o} &  & E \arrow{dd}{p} \\
\\
X \times I \arrow{rr}{G} \arrow[dashed]{uurr}{\widetilde{G}} & & B
\end{tikzcd}
\]
Y una cofibración $i : A \longrightarrow X$ es tal que posee la HEP, que viene dada por el diagrama:
\[
\begin{tikzcd}
	{} 						  & X \drar{i_0} \arrow[bend left]{drr}{f}			   &        					&   \\
	A \urar[hook] \drar{i_0}  &   												   & X \times I \rar[dashed]{F} & Y \\
	   						  & A \times I  \urar[hook] \arrow[bend right]{urr}{G} &   							&
\end{tikzcd}
\]
Éste último diagrama podemos cambiarlo haciendo uso del \textit{currying}, esto es, considerar aplicaciones $X \times I \longrightarrow Y$ como aplicaciones $X \longrightarrow Y^I$. Así, tendríamos que el diagrama queda de la siguiente forma:\[
\begin{tikzcd}
Y & & X \arrow{ll}{f} \arrow[dashed]{ddll}{\widetilde{F}} \\
\\
Y^I \arrow[two heads]{uu}{\pi} & & A \arrow[hook]{uu}{i} \arrow{ll}{G}
\end{tikzcd}
\]
En el cual se ve que el sentido de las flechas es el contrario que en la fibración. \par
Además esto también se observa entre las sucesiones exactas dadas por fibraciones y cofibraciones: \par
Para una cofibración, teníamos la sucesión de Barratt-Puppe:
\[ X \longrightarrow Y \longrightarrow C_f \longrightarrow \Sigma X \longrightarrow \Sigma Y \longrightarrow \Sigma C_f \longrightarrow \Sigma^2 X \longrightarrow \dots \]
Y para las fibraciones teníamos la sucesión dada por la fibra homotópica de una aplicación:
\[ \dots \longrightarrow \Omega^2 Y \longrightarrow \Omega F \longrightarrow \Omega X \longrightarrow \Omega Y \longrightarrow F \longrightarrow X \longrightarrow Y \]

\section{Espacios de Eilenberg-MacLane}
Sea $G$ un grupo y $n$ un entero positivo. Un grupo topológico conexo $X$ se dice que es un espacio de Eilenberg-MacLane de tipo $K(G,n)$ si su grupo de homotopía $n$-ésimo $\homot{n}{X} \cong G$ y $\homot{k}{X} = 0$ si $k \neq n$. Si $n > 1$ entonces $G$ debe ser abeliano como hemos visto.

\begin{ejems}
\begin{enumerate}
\item Espacios de tipo $K(G, 1)$. \par
La circunferencia unidad $S^1$ es un espacio de Eilenberg-MacLane de tipo $K(\bb{Z}, 1)$ y el todo $T = S^1 \times S^1$ es $T = K(\bb{Z} \oplus \bb{Z}, 1)$. También se tiene que $\bb{R}P^\infty = K(\bb{Z}_2, 1)$. \par
En general, un espacio se dice ``asférico'' o de tipo $K(G,1)$ si su recubridor universal $\widetilde{X}$ es débilemente contráctil, esto es, si $\homot{n}{\widetilde{X}} = 0$ $\forall n$.

\item Espacios de tipo $K(G, n)$ con $n \geq 2$.
Como ejemplo de espacio $K(G, 2)$ tenemos al espacio proyectivo complejo $\bb{C}P^\infty = K(\bb{Z}, 2)$. \par
Un ejemplo más general lo tenemos considerando el producto simétrico infinito $SP^\infty (X)$, definido como 
\[
SP^\infty (X) = \faktor{\dot{\cup}_{i \in \bb{N}} X^i}{\sim}
\]
donde $\sim$ es la relación de equivalencia que identifica $(x_1, \ldots, x_n) \sim (x_1, \ldots, x_n, e)$ y cualquier punto con puntos con las mismas coordenadas permutadas. \par
Entonces, el Teorema de Dold-Thom dice que si $X$ es $CW$-complejo, entonces $\homot{i}{SP^\infty (S^n)} \cong \homr{i}{X; \bb{Z}}$ para $i \geq 0$. \par
Así considerando $X = S^n$, 
\[
\homot{i}{SP^\infty (S^n)} = 
\begin{cases}
\bb{Z} & \text{si } i = n \\
0 & \text{resto}
\end{cases}
\]
luego $K(\bb{Z},n) = SP^\infty (S^n)$. \par
Más generalmente, si consideramos el espacio de Moore $M(G, n)$, que no es más que el equivalente homológico al espacio de Eilenberg-MacLane, se tiene que 
\[
K(G, n) = SP^\infty (M(G, n))
\]
\end{enumerate}
\end{ejems}
Además, se puede demostrar lo siguiente:
\begin{teor}
El espacio de Eilenberg-MacLane $K(G, n)$ es único, salvo homotopía.
\end{teor}
\nts{¿Añadir demostración sin demostrar lemas previos?}